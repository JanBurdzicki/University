\documentclass[a4paper,12pt]{article}
\usepackage[utf8]{inputenc}
\usepackage[polish]{babel}
\usepackage[T1]{fontenc}
\usepackage{amsmath}
\usepackage{amssymb}
\usepackage[left = 1.5cm, right = 1.5cm, top = 1.75cm, bottom = 1.75cm]{geometry}

\title{\textbf{PIZZO – Zadanie 2}}
\author{Jan Burdzicki, 339592}
\date{22.11.2024}

\begin{document}

\maketitle

\section{Treść zadania}

Dla programu $X$, który bierze dwie liczby naturalne, coś liczy i zwraca 0 lub 1 (nigdy się nie zapętla), przez $P_X$ oznaczamy problem decyzyjny taki, że $P_X(n)=1$ jeśli istnieje $m$ takie, że $X(n, m)=1$, oraz $P_X(n)=0$ w przeciwnym przypadku. Czy istnieje $X$ taki, że $P_X$ jest nierozstrzygalny?

\section{Rozwiązanie}

Pokażemy przykład programu $X$ takiego, że $P_X$ będzie nierozstrzygalny.

\noindent
Niech $X(n, m)$ będzie programem, który przyjmuje jako argumenty 2 liczby naturalne:
\begin{itemize}
	\item $n$ – numer programu (mamy ponumerowane programy kolejnymi liczbami naturalnymi)
	\item $m$ – ograniczenie na liczbę wykonanych operacji na maszynie Turinga
\end{itemize}

\noindent
Będziemy uruchamiać $n$-ty program, wykonując jego $m$ kolejnych operacji.

\[
X(n, m) =
\begin{cases}
1, & \text{gdy program numer } n \text{ zakończy się po wykonaniu} \le m \text{ operacji}, \\
0, & \text{w p.p.}
\end{cases}
\]

\noindent
Zatem program $X$ spełnia w oczywisty sposób założenia z zadania, czyli:
\begin{itemize}
	\item wejście: 2 liczby naturalne
	\item wyjście: 0 lub 1
	\item nigdy się nie zapętla
\end{itemize}

\noindent
$P_X(n)$ rozumiemy następująco:
\[
P_X(n) =
\begin{cases}
1, & \text{gdy istnieje } m \text{ takie, że program numer } n \text{ zatrzymuje się po} \le m \text{ operacjach}, \\
0, & \text{w p.p.}
\end{cases}
\]

\noindent
Załóżmy nie wprost, że $P_X(n)$ jest rozstrzygalny.
Wtedy dla dowolnego $n$ bylibyśmy w stanie sprawdzić, czy $n$-ty program kiedykolwiek się zatrzyma ($P_X(n) = 1$) lub czy zajdzie zdarzenie przeciwne ($P_X(n) = 0$). Zatem $P_X(n)$ rozstrzygałby problem STOPu. Sprzeczność (problem STOPu jest nierozstrzygalny).

\bigskip

\noindent
Zatem $P_X(n)$ jest nierozstrzygalny.

\end{document}
